\documentclass{article}

\usepackage{amsmath, amsthm, amssymb, amsfonts}
\usepackage{thmtools}
\usepackage{graphicx}
\usepackage{setspace}
\usepackage{geometry}
\usepackage{float}
\usepackage{hyperref}
\usepackage[utf8]{inputenc}
\usepackage[english]{babel}
\usepackage{framed}
\usepackage[dvipsnames]{xcolor}
\usepackage{tcolorbox}

\colorlet{LightGray}{White!90!Periwinkle}
\colorlet{LightOrange}{Orange!15}
\colorlet{LightGreen}{Green!15}

\newcommand{\HRule}[1]{\rule{\linewidth}{#1}}

\declaretheoremstyle[name=Theorem,]{thmsty}
\declaretheorem[style=thmsty,numberwithin=section]{theorem}
\tcolorboxenvironment{theorem}{colback=LightGray}

\declaretheoremstyle[name=Proposition,]{prosty}
\declaretheorem[style=prosty,numberlike=theorem]{proposition}
\tcolorboxenvironment{proposition}{colback=LightOrange}

\declaretheoremstyle[name=Principle,]{prcpsty}
\declaretheorem[style=prcpsty,numberlike=theorem]{principle}
\tcolorboxenvironment{principle}{colback=LightGreen}

\setstretch{1.2}
\geometry{
    textheight=9in,
    textwidth=5.5in,
    top=1in,
    headheight=12pt,
    headsep=25pt,
    footskip=30pt
}

% ------------------------------------------------------------------------------

\begin{document}

% ------------------------------------------------------------------------------
% Cover Page and ToC
% ------------------------------------------------------------------------------

\title{ \normalsize \textsc{}
		\\ [2.0cm]
		\HRule{1.5pt} \\
		\LARGE \textbf{\uppercase{Relatório do trabalho\\ Pet shop}
		\HRule{2.0pt} \\ [0.6cm] \LARGE{Ferramentas da Internet} \vspace*{10\baselineskip}}
		}
\date{}
\author{\textbf{Autores:} \\
		Clarice Cabrini\\
		Hillary Carla\\
            Lara Silva\\
            Maria Eduarda\\
            Sarah Luiza\\
		22/04/2025}

\maketitle
\newpage

% ------------------------------------------------------------------------------

\section{Introdução}

O Pet Shop oferece uma variedade de serviços. Dentre eles banho, tosa e consultas veterinárias, sendo eles executados por funcionários especializados. Os clientes podem agendar os serviços previamente ou por demanda. Cada um desses clientes pode possuir um ou mais animais de estimação, porém cada animal possui apenas e obrigatoriamente um dono. As informações armazenadas dos clientes são nome completo, CPF, e-mail, telefone, data de nascimento e endereço completo. Cada animal possui código de cadastro, nome, espécie, raça, data de nascimento, sexo e peso atual. O sistema registra dados sobre a saúde dos animais. Os serviços oferecidos incluem banho, tosa, consulta, vacinação e aplicação de medicamento. Ele é caracterizado por tipo, descrição, valor base, duração estimada e pode sofrer variação de preço conforme o porte do animal. Os agendamentos dos serviços contém data e hora, status do agendamento, animal a ser atendido, cliente e funcionário. Os funcionários são divididos em diferentes cargos, como banhista, tosador, veterinário e atendente. Cada um deles possui código de identificação, nome completo, telefone, e-mail, cargo e salário. Os veterinários fazem consultas, aplicações de vacina e tratamentos diversos. Cada consulta é vinculada a um animal e um veterinário e possui informações como data, relatório, diagnóstico e prescrição de medicamentos. Cada veterinário possui uma ou mais especialidades. Além de serviço, o Pet Shop realiza venda de produtos como rações, medicamentos, brinquedos e acessórios. Cada produto possuem código, nome, preço, descrição , categoria e validade. Os produtos são fornecidos por fornecedores que possuem identificação, razão social, nome, telefone e CNPJ ou CPF. Um fornecedor fornece vários produtos. Cada Compra é registrada por um código, data da compra, valor total e lista de itens comprados, especificando valor e quantidade.  Cada venda pode envolver produtos ou serviços e é vinculada a um cliente. Ela é identificada por código, data e hora, forma de pagamento, valor total e funcionário responsável pelo atendimento. O sistema registra pagamentos, que podem ser realizados em dinheiro, cartão de crédito ou débito, ou PIX, incluindo informações de parcelamento quando aplicável.

\newpage


\end{document}