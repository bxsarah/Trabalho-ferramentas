\documentclass{article}

\usepackage{amsmath, amsthm, amssymb, amsfonts}
\usepackage{thmtools}
\usepackage{graphicx}
\usepackage{setspace}
\usepackage{geometry}
\usepackage{float}
\usepackage{hyperref}
\usepackage[utf8]{inputenc}
\usepackage[english]{babel}
\usepackage{framed}
\usepackage[dvipsnames]{xcolor}
\usepackage{tcolorbox}

\colorlet{LightGray}{White!90!Periwinkle}
\colorlet{LightOrange}{Orange!15}
\colorlet{LightGreen}{Green!15}

\newcommand{\HRule}[1]{\rule{\linewidth}{#1}}

\declaretheoremstyle[name=Theorem,]{thmsty}
\declaretheorem[style=thmsty,numberwithin=section]{theorem}
\tcolorboxenvironment{theorem}{colback=LightGray}

\declaretheoremstyle[name=Proposition,]{prosty}
\declaretheorem[style=prosty,numberlike=theorem]{proposition}
\tcolorboxenvironment{proposition}{colback=LightOrange}

\declaretheoremstyle[name=Principle,]{prcpsty}
\declaretheorem[style=prcpsty,numberlike=theorem]{principle}
\tcolorboxenvironment{principle}{colback=LightGreen}

\setstretch{1.2}
\geometry{
    textheight=9in,
    textwidth=5.5in,
    top=1in,
    headheight=12pt,
    headsep=25pt,
    footskip=30pt
}

% ------------------------------------------------------------------------------

\begin{document}

% ------------------------------------------------------------------------------
% Cover Page and ToC
% ------------------------------------------------------------------------------

\title{ \normalsize \textsc{}
		\\ [2.0cm]
		\HRule{1.5pt} \\
		\LARGE \textbf{\uppercase{Relatório do trabalho\\ Pet shop}
		\HRule{2.0pt} \\ [0.6cm] \LARGE{Ferramentas da Internet} \vspace*{10\baselineskip}}
		}
\date{}
\author{\textbf{Autores:} \\
		Clarice Cabrini\\
		Hillary Carla\\
            Lara Silva\\
            Maria Eduarda\\
            Sarah Luiza\\
		22/04/2025}

\maketitle
\newpage

% ------------------------------------------------------------------------------

\section{Introdução}
O Pet Shop oferece uma variedade de serviços, dentre eles banho, consulta, vacinação, tratamentos diversos e vendas. Eles são caracterizados por tipo, descrição, valor base e duração estimada. O banho é diferenciado por tosa, a consulta por data e hora, relatório, diagnóstico e prescrição de remédios, a vacinação por peso e a aplicação de remédio por alergias. A venda é diferenciada por código de venda, forma de pagamento, valor total e funcionário responsável e distribui produtos, dentre eles rações, brinquedos, acessórios e medicamentos. Eles possuem código, nome, descrição, preço, categoria e validade. A Ração é destacada por marca, brinquedo por tipo de animal, acessório por porte do animal e pelagem e medicamentos por alergia e peso do animal. Os produtos são fornecidos por fornecedores que possuem CNPJ, razão social e telefone. Os funcionários executam os serviços e são divididos entre banhista, veterinário e atendente. Sua identificação é dada por código único, nome, telefone, cargo e salário. Os banhistas são caracterizados também por nível de experiência, veterinário por CRM e atendente por CPF. Os clientes realizam compras ou agendam serviços previamente ou por demanda, contendo data e hora, status, animal a ser atendido e funcionário. Suas informações são: nome, CPF, e-mail, telefone, data de nascimento e endereço.  Cada um desses clientes possuem um ou mais animais de estimação e os animais possuem obrigatoriamente um dono. Cada animal possui código de cadastro, nome, espécie, raça, idade, sexo e peso. O sistema, caracterizado por data de atendimentos, lucro e feedbacks, registra os animais, pagamentos e as compras. Cada compra é registrada por código, data, valor total e atendente responsável. O pagamento pode ser realizado por dinheiro, pix ou cartão que, se necessário, informa parcelamentos.

\newpage

\section{Descrição dos relacionamentos}

Registra (Pagamento – Sistema) 

Semântica: O sistema registra todos os pagamentos realizados. 

Cardinalidade: Um-para-muitos (um sistema pode registrar vários pagamentos, mas cada pagamento é registrado em apenas um sistema). 

Participação: Parcial para Sistema (um sistema pode existir sem registrar pagamentos). 

Participação: Total para Pagamento (todo pagamento deve ser registrado no sistema). 

Registra (Sistema – Animal) 

Semântica: O sistema registra informações dos animais cadastrados. 

Cardinalidade: Um-para-muitos (um sistema pode registrar vários animais, mas cada animal é registrado em apenas um sistema). 

Participação: Parcial para Sistema (um sistema pode existir sem registrar animais). 

Participação: Total para Animal (todo animal precisa estar registrado no sistema). 

Registra (Sistema – Compra) 

Semântica: O sistema registra todas as compras realizadas. 

Cardinalidade: Um-para-muitos (um sistema pode registrar várias compras, mas cada compra é registrada em apenas um sistema). 

Participação: Parcial para Sistema (um sistema pode existir sem registrar compras). 

Participação: Total para Compra (toda compra precisa estar registrada no sistema). 

Faz (Cliente – Compra) 

Semântica: O cliente realiza compras no sistema. 

Cardinalidade: Um-para-muitos (um cliente pode realizar várias compras, mas cada compra é feita por apenas um cliente). 

Participação: Parcial para Cliente (nem todo cliente realiza compras). 

Participação: Total para Compra (toda compra deve ser feita por um cliente). 

Possui (Cliente – Animal) 

Semântica: O cliente possui animais cadastrados no sistema. 

Cardinalidade: Um-para-muitos (um cliente pode possuir vários animais, mas cada animal pertence a apenas um cliente). 

Participação: Parcial para Cliente (nem todo cliente possui animais). 

Participação: Total para Animal (todo animal deve pertencer a um cliente). 

Agenda (Cliente – Serviço) 

Semântica: O cliente agenda serviços para seus animais. 

Cardinalidade: Muitos-para-muitos (um cliente pode agendar vários serviços, e um serviço pode ser agendado por vários clientes). 

Participação: Parcial para Cliente (nem todo cliente agenda serviços). 

Participação: Parcial para Serviço (nem todo serviço é agendado por clientes). 

Executa (Funcionário – Serviço) 

Semântica: O funcionário executa os serviços agendados. 

Cardinalidade: Muitos-para-muitos (um funcionário pode executar vários serviços, e um serviço pode ser executado por vários funcionários). 

Participação: Parcial para Funcionário (nem todo funcionário executa serviços). 

Participação: Total para Serviço (todo serviço precisa ser executado por pelo menos um funcionário). 

Distribui (Serviço – Venda) 

Semântica: A venda está associada à distribuição de serviços realizados. 

Cardinalidade: Um-para-muitos (um serviço pode originar várias vendas, mas cada venda pertence a apenas um serviço). 

Participação: Parcial para Serviço (nem todo serviço gera venda). 

Participação: Total para Venda (toda venda deve estar ligada a um serviço). 

Fornece (Fornecedor – Produto) 

Semântica: O fornecedor fornece produtos para o sistema. 

Cardinalidade: Um-para-muitos (um fornecedor pode fornecer vários produtos, mas cada produto é fornecido por apenas um fornecedor). 

Participação: Parcial para Fornecedor (nem todo fornecedor fornece produtos no sistema). 

Participação: Total para Produto (todo produto precisa ter um fornecedor). 
\begin{figure}

\section{Dicionário de Entidades}

    \centering
    \includegraphics[width=1\linewidth]{pag1.pdf}
    \caption{Dicionário de entidades}
    \label{fig:placeholder}
\end{figure}

\begin{figure}

\section{Dicionário de Entidades}

    \centering
    \includegraphics[width=1\linewidth]{pag2.pdf}
    \caption{Dicionário de entidades}
    \label{fig:placeholder}
\end{figure}

\begin{figure}

\section{Dicionário de Entidades}

    \centering
    \includegraphics[width=1\linewidth]{pag3.pdf}
    \caption{Dicionário de entidades}
    \label{fig:placeholder}
\end{figure}

\begin{figure}

\section{Dicionário de Entidades}

    \centering
    \includegraphics[width=1\linewidth]{pag5.pdf}
    \caption{Dicionário de entidades}
    \label{fig:placeholder}
\end{figure}


\begin{figure}
\section{Diagrama}
    \centering
    \includegraphics[width=1\linewidth]{novo didi 2.png}
    \caption{Figura - Diagrama Pet Shop}
    \label{fig:placeholder}
\end{figure}




\end{document}